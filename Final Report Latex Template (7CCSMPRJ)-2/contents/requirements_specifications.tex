\section{Requirements \& Specifications}

This section establishes the core technical and functional requirements for developing a SAT solver architecture that prioritises worst-case robustness over average-case performance for moderate-scale graph colouring problems. The requirements address the fundamental research question: \emph{How can graph-aware SAT solver architectures be designed to systematically trade average-case performance for worst-case robustness, ensuring reliable solution capability on challenging instances where baseline approaches fail?} These specifications define reliability-focused objectives and robustness guarantees, whilst detailed implementation approaches and validation methodologies are addressed in subsequent chapters.

\subsection{Robustness-Oriented Functional Requirements}

\subsubsection{Core Reliability Capabilities}
The enhanced SAT solver shall maintain complete DPLL correctness guarantees while incorporating graph-aware robustness enhancements that ensure reliable solution capability on challenging instances. The system must provide configurable robustness modes that systematically trade computational overhead for reliability improvements, including comprehensive preprocessing analysis and fallback mechanisms that guarantee solution capability even when optimisations fail.

The solver shall implement comprehensive error handling and graceful degradation strategies that ensure operational reliability under all conditions. Circuit breaker patterns must automatically disable problematic optimisations while maintaining basic solver functionality, providing robustness guarantees that exceed baseline DPLL implementations. The system shall never perform worse than standard approaches, even when graph-aware optimisations prove ineffective on specific problem instances.

\subsubsection{Comprehensive Graph Analysis for Robustness}
The system shall implement exhaustive graph structural analysis that accepts 15-20\% preprocessing overhead to ensure robust variable ordering on challenging instances. Comprehensive centrality computation must include degree analysis, betweenness centrality for moderate-scale problems, and composite scoring mechanisms that prioritise structural understanding over computational efficiency.

The analysis framework shall provide multiple fallback levels that ensure reliable operation across diverse graph characteristics: sophisticated analysis for challenging instances that justify computational investment, simplified degree-based analysis when computational constraints prevent comprehensive processing, and baseline heuristics when graph analysis encounters failures. This hierarchical approach guarantees robust operation while optimising analysis depth based on problem characteristics and computational resources.

\subsubsection{Reliability-Focused Encoding and Optimization}
The solver shall generate CNF encodings that prioritise correctness and completeness over encoding efficiency, ensuring that robustness optimisations never compromise solution quality or algorithmic guarantees. Direct encoding shall remain the primary strategy with comprehensive validation mechanisms that verify encoding correctness across diverse graph structures.

Symmetry breaking implementation must integrate seamlessly with graph-aware variable ordering while maintaining predictable behaviour characteristics. The system shall employ conservative optimization strategies that prioritize systematic search guidance over aggressive reduction techniques that could introduce unpredictable performance characteristics on challenging instances.

\subsubsection{Trade-off Configuration and Monitoring}
The solver shall provide comprehensive configuration controls that enable systematic evaluation of performance-robustness trade-offs across different operational modes. Users must be able to configure overhead budgets, analysis depth, and fallback thresholds to balance robustness benefits against computational costs based on deployment requirements.

The system shall implement detailed monitoring infrastructure that captures trade-off metrics including preprocessing overhead, robustness improvements, success rate comparisons, and failure mode analysis. This monitoring capability enables systematic validation of trade-off hypotheses while providing deployment guidance for reliability-critical applications.

\subsection{Performance-Robustness Trade-off Requirements}

\subsubsection{Acceptable Overhead Targets}
The enhanced solver shall accept systematic performance overhead in exchange for robustness guarantees, with target overhead factors between 1.2x-2.0x average-case execution time compared to baseline DPLL implementations. Overhead above 2.0x indicates excessive cost requiring optimisation or configuration adjustment, while overhead below 1.2x suggests insufficient robustness investment.

The system must demonstrate predictable overhead characteristics that enable accurate deployment planning and cost-benefit analysis. Performance degradation shall remain consistent across diverse problem characteristics within the target range, avoiding unpredictable behaviour that could undermine reliability for deployment in production systems.

\subsubsection{Reliability Improvement Guarantees}
The enhanced solver must achieve 100\% solution capability on challenging instances where baseline approaches exhibit timeout failures or exponential degradation. Success rate improvements shall be systematic and measurable across diverse graph categories including dense random graphs, high-connectivity instances, and structurally complex problems that stress standard variable ordering heuristics.

Reliability improvements must be validated through comprehensive stress testing that deliberately targets problematic scenarios known to defeat baseline approaches. The system shall provide documented robustness benefits that justify performance overhead through empirical demonstration of enhanced capability on challenging instances.

\subsubsection{Memory and Resource Efficiency}
Peak memory consumption shall scale predictably with robustness analysis complexity, maintaining reasonable resource requirements despite comprehensive preprocessing. Memory overhead associated with graph analysis and priority caching must remain proportional to problem scale and justified by robustness benefits.

The system shall implement adaptive resource management that scales analysis complexity based on available computational budget and problem characteristics. Resource usage must be predictable and configurable to support deployment planning in resource-constrained environments.

\subsection{Robustness Architecture Requirements}

\subsubsection{Fault-Tolerant Modular Design}
The solver shall implement a four-component architecture with comprehensive error handling and fallback capabilities: a robustness-enhanced SAT engine that maintains DPLL guarantees while incorporating graph-aware improvements, a comprehensive graph analysis module that performs structural characterisation with multiple fallback levels, a reliability-focused encoding module that prioritises correctness over efficiency, and a robustness monitoring module that tracks trade-off effectiveness and failure modes.

Each component must implement circuit breaker patterns that automatically disable problematic features while maintaining overall system reliability. Component interfaces shall support independent testing and validation while ensuring that failures in individual modules never compromise overall solver capability.

\subsubsection{Graceful Degradation Specifications}
The system shall implement comprehensive degradation strategies with multiple fallback levels that ensure reliable operation under all conditions. Primary degradation involves disabling sophisticated graph analysis while maintaining enhanced variable ordering capabilities. Secondary degradation reverts to standard DPLL heuristics while preserving enhanced conflict resolution. Final degradation operates identically to baseline DPLL with full compatibility guarantees.

Degradation triggers must be configurable and include timeout detection, memory usage monitoring, analysis failure rates, and performance threshold violations. The system shall maintain detailed logging of degradation events to enable systematic analysis of robustness optimisation effectiveness across diverse operational scenarios.

\subsubsection{Interface Compatibility and Integration}
The enhanced solver shall function as a drop-in replacement for standard DPLL implementations while providing additional robustness configuration options. All existing API contracts must be preserved with no changes to input formats, output specifications, or method signatures that could compromise backward compatibility.

Integration specifications require seamless operation within existing SAT solver testing frameworks and benchmark evaluation systems. The solver shall accept standard SAT input formats while providing additional configuration parameters for robustness optimisation without breaking compatibility with existing infrastructure.

\subsection{Validation and Success Criteria}

\subsubsection{Robustness Validation Benchmarks}
The enhanced solver must demonstrate systematic robustness improvements through comprehensive evaluation on challenging instances that reliably defeat baseline approaches. Success criteria include 100\% solution capability on dense random graphs with edge densities exceeding 0.4 where baseline approaches exhibit timeout failures, consistent performance on high-connectivity instances that stress standard variable ordering heuristics, and reliable operation on structurally complex graphs that expose fundamental limitations in naive search strategies.

Validation methodology shall employ deliberate stress testing that systematically increases problem difficulty until baseline approaches fail, then evaluates enhanced solver behaviour on these challenging instances. This adversarial evaluation approach provides concrete evidence for robustness claims beyond simple performance comparison.

\subsubsection{Trade-off Effectiveness Criteria}
The system must achieve acceptable trade-off ratios that justify performance overhead through measurable reliability improvements. Success criteria include overhead factors between 1.2x-2.0x accompanied by significant success rate improvements, consistent trade-off characteristics across diverse problem categories, and predictable behaviour that enables accurate deployment planning.

Trade-off validation shall employ systematic parameter sweeps across graph characteristics, connectivity patterns, and structural complexity to establish boundaries of optimisation effectiveness. Statistical analysis must demonstrate consistency of trade-off benefits across multiple evaluation dimensions.

\subsubsection{Technical Contribution Requirements}
The project must advance understanding of performance-robustness trade-offs in specialised SAT solving through empirical validation of graph-aware optimisation strategies. Technical contributions shall include demonstrated trade-off quantification with clear cost-benefit analysis, validated robustness improvements on challenging instances with documented failure mode prevention, architectural innovations that enable systematic trade-off evaluation, and methodological advances in reliability-focused solver design.

The implementation must provide novel insights into the relationship between graph structure and SAT solving robustness while maintaining rigorous correctness standards and comprehensive validation protocols. Research contributions shall be validated through peer review and empirical evaluation against established baseline approaches.

\subsection{Scope Limitations and Boundaries}

\subsubsection{Problem Scale and Domain Focus}
The solver development focuses specifically on moderate-scale graph colouring problems within the 50-100 vertex range where comprehensive graph analysis remains computationally feasible while providing measurable robustness benefits. This scope enables detailed investigation of trade-off characteristics while avoiding scalability challenges that could obscure robustness optimisation effectiveness.

The system shall support diverse graph types including regular structures, random graphs with varying density, and pathological instances designed to stress solver robustness. Problem domain boundaries exclude extremely large-scale industrial applications where different optimisation strategies may prove more effective.

\subsubsection{Implementation and Research Boundaries}
The project scope excludes advanced parallel processing optimisations and sophisticated learning mechanisms that could introduce unpredictable behaviour characteristics incompatible with robustness objectives. Extension to related problems such as edge colouring or scheduling applications lies beyond the current scope, maintaining focus on graph colouring optimisation.

The implementation shall prioritise graph-aware robustness enhancements over complex CDCL features, recognising that moderate-scale problems benefit more from improved variable ordering than from sophisticated clause learning mechanisms. Research boundaries maintain focus on trade-off characterisation rather than pursuing absolute performance optimisation.