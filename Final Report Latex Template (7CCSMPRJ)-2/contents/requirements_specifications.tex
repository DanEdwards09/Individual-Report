\section{Requirements \& Specifications}

This section establishes the core technical and functional requirements for developing a SAT solver architecture specifically optimised for moderate-scale graph colouring problems. The requirements address the fundamental research question: \emph{How can SAT solver architectures be specialised and optimised specifically for moderate-scale graph colouring problems to achieve superior performance compared to both generic SAT approaches and traditional graph colouring algorithms?} These specifications define what the system must accomplish, whilst detailed implementation approaches and validation methodologies are addressed in subsequent chapters.

\subsection{Functional Requirements}

\subsubsection{Core Solver Capabilities}
The enhanced SAT solver shall maintain complete DPLL functionality including unit propagation, chronological backtracking, and conflict detection whilst incorporating graph-aware enhancements. The system must support configurable variable ordering strategies that exploit graph structural properties such as vertex degree distribution, saturation levels, and centrality measures. The solver shall seamlessly integrate graph-theoretic heuristics without compromising fundamental correctness guarantees of the underlying SAT solving procedure.

\subsubsection{Graph Processing and Analysis}
The system shall implement comprehensive graph structure analysis including degree distribution computation, connectivity pattern recognition, and vertex centrality calculation. The solver must support multiple standard graph input formats including DIMACS graph format, edge list representation, and adjacency matrix notation, with robust error handling and format validation. Graph preprocessing shall extract structural features that guide optimisation strategies whilst completing within reasonable computational bounds.

\subsubsection{Encoding and Optimisation}
The solver shall generate optimised CNF encodings from graph structures, with direct encoding as the primary strategy and support for alternative encoding approaches where beneficial. The system must implement symmetry breaking for colour permutation symmetries through lexicographic ordering constraints. At-most-one constraints shall utilise efficient encoding techniques that minimise clause count whilst preserving logical completeness and correctness.

\subsubsection{Configuration and Extensibility}
The solver shall provide configurable control over optimisation strategies, enabling users to toggle between graph-aware and generic heuristics for comparative analysis. The system must support parameter configuration for different graph types and problem scales whilst maintaining sensible defaults. The architecture shall enable incremental enhancement through modular design with standardised interfaces for adding new heuristics and encoding strategies.

\subsection{Performance Requirements}

\subsubsection{Execution Time Targets}
The solver shall achieve specific performance targets calibrated for moderate-scale problems on the target hardware environment (Apple M4 MacBook with 16GB RAM). For 50-vertex problems, the system must successfully solve 95\% of instances within 10 seconds. For 75-vertex problems, 75\% of instances shall be solved within 60 seconds. For 100-vertex problems, 50\% of instances must be solved within 300 seconds. The enhanced solver shall demonstrate at least 2× speedup compared to baseline generic DPLL implementation on graph colouring benchmarks.

\subsubsection{Memory and Scalability Constraints}
Peak memory consumption shall remain under 4GB for 100-vertex problems, ensuring compatibility with target hardware constraints. CNF formula size growth must maintain near-linear relationship with graph size, avoiding exponential memory consumption that would compromise scalability. The system shall exhibit predictable performance degradation as graph size increases, avoiding sudden performance cliffs within the target range of 50-100 vertices.

\subsubsection{Quality and Reliability}
The solver must generate correct solutions that satisfy all graph colouring constraints, with no adjacent vertices sharing identical colours. The system shall provide consistent results across multiple runs of identical problem instances when using controlled conditions. Solution quality must match or exceed existing approaches within the target problem domain, with particular emphasis on minimising the number of colours used where computationally feasible.

\subsection{System Architecture Requirements}

\subsubsection{Modular Design Structure}
The solver shall be built using four separate, independent modules that work together: a core SAT engine that handles the main solving logic, a graph analysis module that examines graph properties and structure, an encoding module that converts graph problems into logical formulas, and a heuristics module that makes intelligent decisions based on graph characteristics. Each component must expose well-defined interfaces enabling independent testing and modification without affecting other system components.

\subsubsection{Interface and Integration Specifications}
Components must communicate through standardised interfaces for performance monitoring and flexible selection mechanisms for choosing algorithms, ensuring modules remain independent and well-organised throughout the system. The SAT engine interface shall accept graph-aware variable ordering strategies and preprocessing directives whilst maintaining compatibility with standard DPLL procedures. Solution output must provide both internal SAT assignment format and human-readable graph colouring representation.

\subsection{Technical Constraints and Scope}

\subsubsection{Problem Scale and Domain}
The solver development focuses specifically on moderate-scale graph colouring problems within the 50-100 vertex range. This scope enables detailed investigation of graph-structural optimisations that provide measurable advantages over generic approaches whilst remaining computationally feasible for comprehensive evaluation. The system shall support diverse graph types including regular structures, random graphs, and real-world network topologies within this scale range.

\subsubsection{Implementation Boundaries}
The project scope excludes advanced parallel processing optimisations and integration with external graph analysis libraries to maintain focus and feasibility within academic constraints. Extension to related problems such as edge colouring or list colouring lies beyond the current scope. The implementation shall prioritise graph-aware optimisations over complex learning mechanisms, recognising that sophisticated clause learning may provide diminishing returns at moderate problem scales.

\subsection{Success Criteria}

\subsubsection{Performance Benchmarks}
The enhanced solver must demonstrate measurable superiority over baseline approaches through multiple evaluation dimensions including solution time improvement, success rate enhancement, and memory efficiency gains. Performance improvements must be consistent across diverse benchmark instances within the target scale range. The system shall outperform both generic SAT solvers and traditional graph colouring approaches on representative problem sets.

\subsubsection{Technical Contributions}
The project must contribute novel insights into the relationship between graph structure and SAT solving efficiency, with demonstrated improvements in algorithm design for moderate-scale combinatorial problems. Technical innovations should include effective graph-aware heuristics, efficient encoding strategies, or architectural improvements applicable to related optimisation domains. The implementation must advance understanding of specialised SAT solving techniques whilst maintaining rigorous correctness standards.