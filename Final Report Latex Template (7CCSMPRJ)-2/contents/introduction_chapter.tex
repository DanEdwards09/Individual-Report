\section{Introduction}

\subsection{Problem Context \& Motivation}

The graph colouring problem stands as one of the most fundamental and computationally challenging problems in discrete mathematics and computer science, requiring the assignment of colours to graph vertices such that no two adjacent vertices share the same colour while minimising the total number of colours used. Although it may seem simple at first, this optimisation problem possesses profound complexity. Classified as non-deterministic polynomial-time complete (NP-complete), it demonstrates exceptional versatility across diverse application domains within modern computing systems.

Real-world applications of graph colouring span critical infrastructure and commercial systems at various scales. In academic scheduling applications, moderate-scale graphs representing 50-100 courses or examination slots require efficient colouring solutions to eliminate conflicts and optimise resource utilisation. Compiler optimisation uses graph colouring to assign computer variables to processor storage locations, where graphs of similar scale represent variables that cannot share the same storage at the same time. Local telecommunications networks employ graph colouring for frequency assignment among cell towers, ensuring adjacent transmitters operate on different frequencies to minimise interference within regional coverage areas.

Current computational approaches face interesting trade-offs when addressing moderate-scale graph colouring problems. Traditional heuristic methods, while computationally efficient, frequently fail to guarantee optimal solutions and may produce suboptimal colourings that waste resources in precisely those applications where optimisation matters most. Exact algorithms based on backtracking or branch-and-bound methodologies can handle instances with 50-100 vertices but often employ generic search strategies that ignore valuable graph-structural information. This represents a research opportunity, as moderate-scale problems can be solved within reasonable time limits while offering sufficient complexity to benefit from specialised algorithmic approaches that leverage graph properties effectively.

\subsection{SAT Solving Background \& Technological Context}

Boolean satisfiability (SAT) solving has transformed computational problem-solving, providing a unified framework for addressing diverse combinatorial optimisation challenges. Modern SAT solvers represent sophisticated software systems incorporating decades of algorithmic innovation, with Conflict-Driven Clause Learning (CDCL) engines forming the foundation of current solving architectures that enable intelligent exploration of exponentially large solution spaces.

Contemporary SAT solver architectures integrate several advanced techniques that collectively achieve high performance across problem scales. The Variable State Independent Decaying Sum (VSIDS) heuristic provides adaptive variable ordering that responds dynamically to problem structure and search progress. Conflict-driven clause learning mechanisms extract and permanently record conflict information, preventing repeated exploration of infeasible solution regions. While restart strategies periodically abandon current search paths to escape local optima, clause deletion policies manage memory consumption during extended solving processes.

For moderate-scale combinatorial problems, SAT solving presents unique opportunities and challenges. While industrial-scale SAT instances with millions of variables benefit primarily from sophisticated learning and search strategies, problems with 50-100 variables offer different optimisation potential. At this scale, the overhead of complex learning mechanisms may not provide proportional benefits, while graph-structural properties have the potential to be more influential in determining search efficiency. The translation of graph colouring problems into SAT instances enables direct application of established solving techniques while creating opportunities for problem-specific optimisations that exploit graph structure.

This technological landscape suggests that specialised SAT solving approaches, tuned specifically for moderate-scale graph colouring problems, could achieve higher performance compared to both generic basic SAT solvers and traditional graph colouring algorithms by focusing on graph-aware optimisations rather than general-purpose search sophistication.

\subsection{Research Objectives \& Commercial Significance}

This project addresses the fundamental research question: \textit{How can SAT solver architectures be specialised and optimised specifically for moderate-scale graph colouring problems to achieve superior performance compared to both generic SAT approaches and traditional graph colouring algorithms?} The investigation is guided by the hypothesis that incorporating graph-aware heuristics, specialised encoding strategies, and structural preprocessing into SAT solver design will yield measurable improvements in solution quality and computational efficiency for problems with approximately 100 vertices.

The primary technical objectives involve developing a SAT solver architecture specifically optimised for moderate-scale graph colouring problems, implementing decision heuristics that exploit graph structural properties such as vertex degree distribution, connectivity patterns, and local density, creating encoding strategies that efficiently represent colouring constraints while minimising clause overhead, and integrating symmetry-breaking mechanisms that reduce search space complexity without compromising solution completeness.

Performance optimisation goals include achieving superior runtime performance compared to general-purpose SAT solvers when applied to graph colouring benchmarks in the 100 vertex range, demonstrating consistent solution quality across diverse graph types including regular structures, random graphs, and real-world network topologies, reducing computational resource requirements while maintaining optimality guarantees, and providing robust performance characteristics that scale predictably within the target problem size range.

The commercial and societal implications of this research extend beyond academic interest, particularly for applications operating at moderate scales. Improved graph colouring algorithms can directly enhance scheduling system efficiency in educational institutions managing course conflicts, optimise register allocation in compiler design for embedded systems with limited resources, and reduce interference in local wireless communication networks. These improvements translate into measurable economic benefits through reduced computational costs, improved resource utilisation, and enhanced system performance across multiple industries operating at regional or institutional scales.

\subsection{Technical Scope \& Methodological Approach}

The project scope contains comprehensive development of specialised SAT solving techniques tailored specifically for moderate-scale graph colouring optimisation. Core technical components include strategic enhancements to CDCL engine architecture that emphasise graph-aware decision making over complex learning mechanisms, implementation of graph-structural preprocessing algorithms that analyse connectivity patterns and vertex properties before encoding generation, development of encoding strategies optimised for different graph types within the target size range, and integration of symmetry-breaking techniques that effectively reduce problem complexity for moderate-scale instances.

The methodological approach emphasises rigorous experimental validation through a systematic analysis of the relationship between graph structural properties and solver performance characteristics, and detailed investigation of the trade-off between encoding complexity and search efficiency at moderate scales.

The focus on moderate-scale problem instances enables deep investigation of graph-structural optimisations that might be overshadowed by search complexity in larger instances. This scale represents a significant practical domain where graph-aware algorithmic enhancements can provide measurable advantages over both generic SAT approaches and traditional heuristic methods, while remaining computationally feasible for comprehensive experimental evaluation and detailed performance analysis.

However, the scope of the project excludes certain optimisation directions to maintain focus and feasibility within academic constraints. Advanced parallel processing optimisations, while potentially beneficial, are beyond the current scope, as is integration with external graph analysis libraries or extension to related problems such as edge colouring, list colouring, or problems requiring significantly larger scale optimisation.

\subsection{Report Organisation \& Contribution Framework}

This report provides comprehensive documentation of the specialised SAT solver development process, starting with thorough analysis of existing literature in both SAT solving and graph-colouring domains with particular attention to moderate-scale optimisation strategies, proceeding with detailed specification of solver requirements and architectural design decisions optimised for the target problem scale, comprehensive documentation of implementation approaches highlighting novel contributions in graph-aware optimisation techniques, rigorous presentation of testing methodologies and experimental validation procedures across various moderate-scale benchmarks, and concluding with detailed evaluation of results and analysis of implications for specialised combinatorial optimisation.

The contribution framework emphasises clear distinction between established SAT solving techniques adapted for moderate-scale problems and novel graph-aware innovations developed specifically for this research, ensuring compliance with academic standards while highlighting the original research contributions that advance understanding of specialised optimisation for structured combinatorial problems within computationally tractable problem scales.